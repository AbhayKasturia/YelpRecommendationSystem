\documentclass[11pt]{article}
%%%%%%%%%%%%%%%%%%%%%%%%%%%%%%%%%%%%%%%%%

\usepackage{amscd}
\usepackage{amsmath}
\usepackage{amssymb}
\usepackage{amsthm}


\usepackage{epsfig}
\usepackage{verbatim}
\usepackage{graphicx}
\usepackage{amsthm}
\usepackage{multirow}
\usepackage{hyperref}
\pagestyle{empty}
\usepackage{color}
\usepackage[left=3cm,top=2.5cm,right=3cm,bottom=3cm]{geometry} % Document margins
%\usepackage[all,dvips]{xy}


\begin{comment}  

This LaTeX document is a template to be used by Bates mathematics rising seniors to create a thesis proposal. 

As a guide, the document is already filled out to represent a fictitious proposal, and all you need to do is modify the entries below to represent your own proposal.

A PDF version of the fictitious proposal is available on the department's FAQ and Policies pages, at
http://abacus.bates.edu/acad/depts/math/faq.html
and
http://abacus.bates.edu/acad/depts/math/policies.html
respectively.

Once you have finished your proposal, export it to a PDF file. Give the file a USEFUL name, for example, RiemannThesisProposal.PDF. Email the PDF file to Clementine Brasier, the 
Academic Administrative Assistant for Hathorn Hall, at cbrasier\@bates.edu

This LaTex document was created Feb/Mar 2010 by Adriana Salerno and updated Feb 2012 by Meredith Greer

\end{comment}


%%%%%%%%%%%%%%%%%%%%%%%%%%%%%%%%%%%%%%%%%

\begin{document}
	\begin{titlepage}
		\centering
		\includegraphics[width=0.15\textwidth]{NU_logo.png}\par\vspace{1cm}
		{\scshape\LARGE Northeastern University \par}
		\vspace{1cm}
		{\scshape\Large Data Mining Project Report (Draft)\par}
		\vspace{1.5cm}
		{\huge\bfseries Recommendation Systems for Yelp Dataset\par}
		\vspace{2cm}
		{\Large\itshape Aditya Priyadarshi, Abhay Kasturia, Xingxing Liu, Varun Nandu and Gautam Vashisht \par}
		%{\Large\itshape (priyadarshi.a@husky.neu.edu) \par}
		\vfill
		Supervised by\par
		Dr. Nate \textsc{Derbinsky}
		
		\vfill
		
		% Bottom of the page
		{\large \today\par}
	\end{titlepage}
	
	\section{Introduction and Related Work}
	
	
	\bigskip
	
	
	\section{Dataset and Analysis} 
		\subsection{Dataset}
		 
		
		\subsection{Analysis}
		
		
	\section{Methods}
		
		\subsection{Clustering Based Approach}
		
		\subsection{Collaborative Filtering}
		
		\subsection{Collaborative Deep Learing}
		
		
		
	
	\section{Experiments and Results}
	
	\section{Future work}
	
	
	\begin{thebibliography}{9}
		% NOTE: change the "9" above to "99" if you have MORE THAN 10 references.
		
	\bibitem{yelp} Yelp Dataset Challenge \url{https://www.yelp.com/dataset_challenge}
	\bibitem{cfilter}Collaborative filtering \url{https://en.wikipedia.org/wiki/Collaborative_filtering}
	\bibitem{ydeep} Paul Covington, Jay Adams, Emre Sargin \textit{Deep Neural Networks for YouTube Recommendations}, ACM 2016.
		
	\end{thebibliography}
	%%%%%%%%%%%%%%%%%%%%%%%%%%%%%%%%%%%%%%%%%
	
	
	
	
	
	
\end{document} 
